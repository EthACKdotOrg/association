\documentclass[12pt,twoside]{report}

\usepackage{geometry}
 \geometry{
 a4paper,
 total={170mm,257mm},
 left=20mm,
 top=20mm,
 }

\usepackage[T1]{fontenc}
\usepackage{libertine}



\usepackage[utf8]{inputenc}
\usepackage[french]{babel}
\usepackage{graphicx}
\usepackage{hyperref}
\usepackage{lastpage}
\usepackage{fancyhdr}
 \fancypagestyle{plain}{%
 \fancyhf{}%
 \fancyfoot[L]{The Swiss Privacy Basecamp}
 \fancyfoot[C]{\thepage/\pageref*{LastPage}}%
 \fancyfoot[R]{EthACK.org — Règlement}%
 \renewcommand{\headrulewidth}{0pt}%
 \renewcommand{\footrulewidth}{0pt}%
 }

\author{EthACK.org — The Swiss Privacy Basecamp}
\title{Règlement interne}
{
\hypersetup{
	colorlinks=true,
	linkcolor=black,
	urlcolor=blue
}
\begin{document}
\begin{titlepage}
\centering
\includegraphics[width=0.50\textwidth]{../logo-4096.png}\par\vspace{2cm}
{\scshape\LARGE EthACK.org \par}
\vspace{1cm}
{\scshape\Large The Swiss Privacy Basecamp \par}
\vspace{1.5cm}
{\huge\bfseries Règlement interne\par}

\vfill
{\large \today\par}
\end{titlepage}

\tableofcontents

\chapter*{Cahier des charges pour les postes}
\addcontentsline{toc}{chapter}{Cahier des charges pour les postes}
\section*{Comité}
\addcontentsline{toc}{section}{Comité}

\subsection*{Conseiller/ère juridique}
Se renseigne et informe le Comité sur tous les aspects légaux des activités prévues ;

\subsection*{IT}
Gère des ressources informatiques ;

\subsection*{Président/e}
\begin{itemize}
\item conduite des Assemblées Générales et du Comité ;
\item représentation légale de l’Association ;
\item possède une carte d’accès au compte ;
\end{itemize}

\subsection*{Relations publiques}
\begin{itemize}
\item gestion des comptes sur les réseaux sociaux ;
\item recherche de contacts ;
\item recherche de sponsors et soutiens ;
\item relation médias ;
\end{itemize}

\subsection*{Secrétaire}
\begin{itemize}
\item prise des procès-verbaux des différentes assemblées ;
\item gestion du courrier entrant et sortant ;
\item convocation des membres aux différentes assemblées ;
\item gestion des communications entre le Comité et les membres ;
\item gestion des membres ;
\end{itemize}

\subsection*{Trésorier/ère}
\begin{itemize}
\item gestion des comptes et établissement des rapports de comptes ;
\item gestion des cotisations ;
\item établissement des factures ;
\item paiement des factures et défraiements ;
\end{itemize}

\section*{Autres fonctions}
\addcontentsline{toc}{section}{Autres fonctions}

\subsection*{Conférencier/ère}
\begin{itemize}
\item validé par le Comité ;
\item préparation des supports de conférence selon le template existant ;
\item déplacement et présentation dans les différents lieux ;
\item reverse les gains potentiels sur le compte de l’association dans les dix jours ouvrés, ou les transmets à un détenteur de la carte, et avise le/a trésorier/ère ;
\item vient chercher le matériel de présentation, en prend soin et le rend en état dans les dix jours ouvrés ;
\item reçoit un accès personnel au dépôt des présentations et le gère de manière responsable ;
\end{itemize}

\subsection*{Membre actif}
\begin{itemize}
\item peut ajouter des articles sur le site (un système de validation est mis en place par le Comité) ;
\item peut prétendre à être conférencier/ère ;
\end{itemize}

\chapter*{Règlement sur les frais}
\addcontentsline{toc}{chapter}{Règlement sur les frais}

Aucun défraiement ne sera fait durant la première année.

Les notes de frais sont validées par le/a Président/e et le/a Trésorier/ère.

\section*{Comité}
\addcontentsline{toc}{section}{Comité}

Les séances de comité ne donnent droit à aucun défraiement à l’exception des frais effectifs éventuels hors transports.

\section*{Autres membres}
\addcontentsline{toc}{section}{Autres membres}
\begin{itemize}
\item Le/a conférencier/ère reçoit un défraiement forfaitaire de 15CHF (quinze francs suisses) ;
\item Les supports de présentation (projecteur, rallonges) sont fournis par l’Association ;
\end{itemize}

\chapter*{Autonomie financière}
\addcontentsline{toc}{chapter}{Autonomie financière}
Toute dépense hors budget au-delà de 100CHF (cent francs suisses) doit être validée par le Comité.

\end{document}