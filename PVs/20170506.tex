\documentclass[12pt,twoside]{report}

\usepackage{geometry}
 \geometry{
 a4paper,
 total={170mm,257mm},
 left=20mm,
 top=20mm,
 }

\usepackage[T1]{fontenc}
\usepackage{libertine}
\usepackage[utf8]{inputenc}
\usepackage[french]{babel}
\usepackage{graphicx}
\usepackage{hyperref}
\usepackage{lastpage}
\usepackage{fancyhdr}
 \fancypagestyle{plain}{%
 \fancyhf{}%
 \fancyfoot[L]{The Swiss Privacy Basecamp}
 \fancyfoot[C]{\thepage/\pageref*{LastPage}}%
 \fancyfoot[R]{EthACK.org — Séance du 06.05.2017}%
 \renewcommand{\headrulewidth}{0pt}%
 \renewcommand{\footrulewidth}{0pt}%
 }

\author{EthACK.org — The Swiss Privacy Basecamp}
\title{Assemblée Générale Ordinaire}
{
\hypersetup{
	colorlinks=true,
	linkcolor=black,
	urlcolor=blue
}
\begin{document}
\begin{titlepage}
\centering
\includegraphics[width=0.50\textwidth]{../logo-4096.png}\par\vspace{2cm}
{\scshape\LARGE EthACK.org \par}
\vspace{1cm}
{\scshape\Large The Swiss Privacy Basecamp \par}
\vspace{1.5cm}
{\huge\bfseries Assemblée Générale\par}

\vfill
{\large \today\par}
\end{titlepage}
\setlength{\parindent}{0cm}

\subsection*{Ouverture}
La séance est ouverte à 10:30 avec un apérifif.


\subsection*{Membres présents}
\begin{itemize}
\item Cédric Jeanneret - Président
\item Cédric Urech - Trésorier
\item Marc Savioz - Secrétaire
\item Diego Abelenda
\item Yves Bertino
\end{itemize}

\subsection*{Revue de l'année 2017}
Le P.V. de l'assemblée générale de 2017 est lu.

L'association a donné plusieurs conférences durant l'année précédente, cela a permis de sensibiliser un public large aux questions de vie privée et de protection des données personnelles.

Les membres discutent du manque de publications sur le site internet. Il faudrait plus d'activité afin de toucher un public plus large

Les comptes sont lus, il faudra les valider. Le comité exprime son envie d'avoir deux membres pour les postes de vérificateurs des comptes. Certains membres proposent de rester sans contrôleurs et d'en élire quand l'association sera plus grande.

Les cotisations ont étées, de manière générale, payées rapidement. 

L'association compte actuellement 10 membres, dont 2 membres n'ayant pas payé leur cotisation. Il faudra voter sur leur exclusion.

Le trésorier note toutefois que des rappels ont du être envoyés par mail et par courrier, et qu'il serait bien que les rappels par courrier ne soient plus nécessaires.


\subsection*{Validation des comptes}

Les comptes sont lus par tous les membres et validés à l'unanimité.

Il y a actuellement 1'407.60 CHF dans les caisses de l'association, dont 329.05 CHF ont été attribués au projet IMSI Catcher.

Il reste donc un montant de 1'078.55 CHF qui n'est pas attribué dans les caisses.


\subsection*{Exclusion des membres n'ayant pas payé leur cotisation}

Les membres n'ayant pas payés leur cotisation sont

\begin{itemize}
\item{Jeremie Fontanaz}
\item{Matthias Brugmann}
\end{itemize}

L'assemblée générale vote leur exclusion à l'unanimité.

\subsection*{Cotisations}

Pour attirer plus de membres, l'assemblée suggère un tarif étudiant. 

Le trésorier explique qu'actuellement les comptes sont sains, et qu'une offre moins chère ne mettrait pas en danger la trésorerie.

Cédric J. propose un montant annuel de 50 CHF pour les étudiants, sur présentation d'une carte de légitimation.

Les membres présent à l'assemblée sont tous d'accord sur ce point, il est donc validé.

\subsection*{Projets}

Les membres font une liste des projets qui pourraient être réalisés dans l'année à venir. Notamment:

\begin{itemize}
\item{IMSI Catcher}
\item{Instance Mastodon}
\item{Diaspora}
\item{Articles sur le blog}
\item{Attirer des compétences}
\end{itemize}

\subsubsection*{IMSI Catcher}

Le projet IMSI catcher est le projet principal sur lequel nous travaillons actuellement, note le président. Le président explique l'objectif qui est de sensibiliser les gens aux risques liés aux attaques possibles sur le réseau mobile.

Ce projet nécessite l'achat de matériel spécialisé et de compétences techniques avancée. Il est aussi noté qu'il serait intéressant de publier des articles techniques sur le blog afin de permettre à plus de chercheurs de travailler sur le sujet des failles du réseau mobile.

\subsubsection*{Mastodon}

Le président explique qu'il serait potentiellement intéressant de créer notre propre instance Mastodon, un clone de Twitter, afin de pouvoir avoir accès à un réseau social décentralisé, et de participer à son développement.


\subsubsection*{Diaspora}

Parallèlement à Mastodon, le président explique qu'il pourrait être intéressant de créer notre instance Diaspora, une alternative à Facebook.


\subsubsection*{Articles}

L'assemblée note qu'il faudrait publier plus fréquemment des articles sur le site internet, car actuellement l'activité est très faible. Une activité plus élevée permettrai de nous rendre plus visibles au grand public  ainsi qu'aux médias.

\subsubsection*{Attirer des compétences}
L'assemblée note qu'il faudrait aussi attirer des personnes aux compétences variées (droit, technique, etc.) dans l'association afin de permettre la publication d'articles plus complets. 

\pagebreak

\subsection*{Questions et Remarques}

Les membres de l'assemblée posent les questions suivantes:

\subsubsection*{Flyers}
Le président explique que des flyers pourraient êtres distribués lors des conférences que l'association donne, afin de mieux familiariser le public avec notre organisation.

Il faut trouver un imprimeur. Le président s'en occupe.

\subsubsection*{Site}

Un membre suggère de mettre à jour le design du site, car actuellement il ne lui parait pas très attractif. 

La majorité des membres est d'accord sur ce point, et il faut que les membres développent un design plus moderne.


\subsection*{Clôture de l'assemblée}

L'assemblée est close à 13:30


\end{document}
