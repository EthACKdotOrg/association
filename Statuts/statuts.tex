\documentclass[12pt,twoside]{report}

\usepackage{geometry}
 \geometry{
 a4paper,
 total={170mm,257mm},
 left=20mm,
 top=20mm,
 }

\usepackage[T1]{fontenc}
\usepackage{libertine}



\usepackage[utf8]{inputenc}
\usepackage[french]{babel}
\usepackage{graphicx}
\usepackage{hyperref}
\usepackage{lastpage}
\usepackage{fancyhdr}
 \fancypagestyle{plain}{%
 \fancyhf{}%
 \fancyfoot[L]{The Swiss Privacy Basecamp}
 \fancyfoot[C]{\thepage/\pageref*{LastPage}}%
 \fancyfoot[R]{EthACK.org — Statuts}%
 \renewcommand{\headrulewidth}{0pt}%
 \renewcommand{\footrulewidth}{0pt}%
 }

\author{EthACK.org — The Swiss Privacy Basecamp}
\title{Statuts de l'Association}
{
\hypersetup{
	colorlinks=true,
	linkcolor=black,
	urlcolor=blue
}
\begin{document}
\begin{titlepage}
\centering
\includegraphics[width=0.50\textwidth]{../logo-4096.png}\par\vspace{2cm}
{\scshape\LARGE EthACK.org \par}
\vspace{1cm}
{\scshape\Large The Swiss Privacy Basecamp \par}
\vspace{1.5cm}
{\huge\bfseries Statuts de l'Association\par}

\vfill
{\large \today\par}
\end{titlepage}

\tableofcontents

\setlength{\parindent}{0cm}

\chapter*{Dénomination et Siège}
\addcontentsline{toc}{chapter}{Dénomination et Siège}

\section*{Article 1}
EthACK.org est une Association sans but lucratif régie par les présents statuts et subsidiairement par les articles 60 et suivants du Code civil suisse\footnote{\url{http://www.admin.ch/ch/f/rs/210/index1.html}}. Elle est politiquement neutre et confessionnellement indépendante.

\section*{Article 2}
Le siège de l’Association est situé à l’adresse du président.

\chapter*{Buts}
\addcontentsline{toc}{chapter}{Buts}

\section*{Article 3}
L’Association poursuit les buts suivants :
\begin{itemize}
\item informer le public de leurs "droits numériques" via, par exemple, des articles en ligne, dans la presse ou des conférences ;
\item informer le public des possibles atteintes par des privés ou un état à ces droits via les mêmes moyens ;
\item promouvoir les outils servant à protéger sa sphère privée ;
\item former et sensibiliser le public pour protéger sa sphère privée ;
\item défendre les droits numériques actuels, et les renforcer ;
\end{itemize}

Les activités de l’Association doivent être conformes au droit applicable.

\chapter*{Ressources}
\addcontentsline{toc}{chapter}{Ressources}

\section*{Article 4}
Les ressources de l’Association proviennent au besoin :
\begin{itemize}
\item de dons et legs ;
\item du parrainage ;
\item de subventions publiques ou privées ;
\item des cotisations versées par les membres ;
\item de toute autre ressource autorisée par la loi ;
\end{itemize}

Les comptes sont publiés après acceptation par l’Assemblée Générale. L’origine des revenus sera nominativement indiquée.

\chapter*{Membres}
\addcontentsline{toc}{chapter}{Members}

\section*{Article 5}
Peut être membre toute personne ou organisme intéressé/e par les buts de l’Association.

\section*{Article 6}
L’Association est composée de :
\begin{itemize}
\item Membres Fondateurs ;
\item Membres Actifs ;
\item Membres de Soutien ;
\end{itemize}

La cotisation des Membres Fondateurs est fixée par l’Assemblée Générale.

Les Membres Actifs participent à la vie de l’Association par leur implication dans les différents buts définis à l’Article 3. La cotisation est fixée par l’Assemblée Générale.

La cotisation des Membres de Soutien est fixée par l’Assemblée Générale. Ils ne peuvent représenter l’Association dans des événements.


\section*{Article 7}
Les demandes d’admission sont adressées au Comité. Le Comité admet ou refuse les nouveaux membres et en informe l’Assemblée Générale.

\section*{Article 8}
La qualité de membre se perd :
\begin{itemize}
\item par la démission. Dans tous les cas la cotisation de la période en cours reste due ;
\item par l’exclusion pour de justes motifs à la majorité du comité ;
\item par décès ;
\item par défaut de paiement des cotisations pendant plus d'une année ;
\end{itemize}

Un Membre Fondateur ne peut être exclu qu’à la majorité des Membres Fondateurs actifs dans l’Association.

\section*{Article 9}
Le patrimoine de l'Association répond seul aux engagements contractés en son nom. Toute responsabilité personnelle de ses membres est exclue.

\chapter*{Organes}
\addcontentsline{toc}{chapter}{Organes}

\section*{Article 10}
Les organes de l’Association sont :
\begin{itemize}
\item L’Assemblée Générale.
\item Le Comité.
\item Les Membres Fondateurs.
\item L’organe de contrôle des comptes.
\end{itemize}

Tant qu’un Membre Fondateur est actif dans l’Association, au moins un Membre Fondateur doit être au Comité.

\chapter*{Assemblée Générale}
\addcontentsline{toc}{chapter}{Assemblée Générale}

\section*{Article 11}
L’Assemblée Générale est le pouvoir suprême de l’Association. Elle est composée de tous les membres.

Elle se réunit une fois par an en session ordinaire. Elle peut, en outre, se réunir en session extraordinaire chaque fois que nécessaire à la demande du Comité, des Membres Fondateurs ou d’un cinquième des membres.

L’Assemblée Générale est valablement constituée quel que soit le nombre de membres présents.

Le Comité communique aux membres par écrit la date de l'Assemblée générale au moins 3 semaines à l'avance. La convocation mentionnant l'ordre du jour est adressée par le Comité à chaque membre au moins 10 jours à l'avance.

\section*{Article 12}
L’Assemblée Générale :
\begin{itemize}
\item élit les membres du Comité et désigne au moins un/e Président/e, un/e Secrétaire et un/e Trésorier/ère ;
\item prend connaissance des rapports et des comptes de l’exercice et vote leur approbation ;
item approuve le budget annuel ;
\item fixe le montant des cotisations annuelles ;
\item prend position sur les autres sujets portés à l’ordre du jour ;
\item adopte et modifie les statuts ;
\end{itemize}

\section*{Article 13}
Les décisions de l’Assemblée Générale sont prises à la majorité simple des membres actifs présents. En cas d’égalité des voix, celle du président est prépondérante.
Les votations ont lieu à main levée. À la demande d’au moins cinq membres, elles auront lieu à bulletin secret.

Il n’y a pas de vote par procuration.

Les Membres de Soutien n’ont pas le droit de vote, et seuls les Membres Actifs à jour de cotisation peuvent voter.

\section*{Article 14}
L’Assemblée Générale est présidée par le président ou un membre du Comité.

\section*{Article 15}
L’ordre du jour de l’Assemblée Générale annuelle, dite "ordinaire", comprend nécessairement :
\begin{itemize}
\item l’approbation du procès-verbal de l’assemblée précédente ;
\item le rapport du Comité sur l’activité de l’Association durant la période écoulée ;
\item les rapports de trésorerie et de l’organe de contrôle des comptes ;
\item la fixation des cotisations ;
\item l’adoption du budget ;
\item l’approbation des rapports et comptes ;
\item l’élection des membres du Comité et de l’organe de contrôle des comptes ;
\item les propositions individuelles ;
\end{itemize}

\chapter*{Comité}
\addcontentsline{toc}{chapter}{Comité}

\section*{Article 16}
Le Comité est autorisé à faire tous les actes qui se rapportent au but de l'association. Il a les pouvoirs les plus étendus pour la gestion des affaires courantes.

\section*{Article 17}
Le Comité se compose au minimum de 3 membres élus par l’Assemblée Générale.

La durée du mandat est de deux ans, renouvelable quatre fois. Les Membres Fondateurs n’ont pas de limite de mandat.

Il se réunit autant de fois que les affaires de l'association l'exigent.

\section*{Article 18}
Les membres du comité agissent bénévolement et ne peuvent prétendre qu'à l'indemnisation de leurs frais effectifs et de leurs frais de déplacement.

Les indemnisations sont régulées par le règlement interne de l’Association.

\section*{Article 19}
Le Comité est chargé :
\begin{itemize}
\item de prendre les mesures utiles pour atteindre les objectifs visés ; 
\item de convoquer les assemblées générales ordinaires et extraordinaires ; 
\item de prendre les décisions relatives à l’admission et à la démission des membres ainsi qu’à leur exclusion éventuelle ; 
\item de veiller à l’application des statuts, de rédiger les règlements et d’administrer les biens de l’Association ; 
\item de s'assurer de la bonne marche des moyens de communication de l'association ; 
\item de rédiger le règlement interne ;
\end{itemize}

\section*{Article 20}
L’Association est valablement engagée par la signature collective de deux membres du Comité, dont le président.

Si ce dernier n'est pas disponible rapidement et qu'il y a urgence, deux autres membres du Comité peuvent signer ensemble, à charge pour eux d'informer le président dès que possible. 

\chapter*{Dispositions diverses}
\addcontentsline{toc}{chapter}{Dispositions diverses}

\section*{Article 21}
L'exercice comptable commence le 1er janvier et se termine le 31 décembre de chaque année.
La gestion des comptes est confiée au trésorier de l'association et contrôlée chaque année par le(s) vérificateur(s) nommé(s) par l'Assemblée Générale.

\section*{Article 22}
En cas de dissolution de l'association, l'actif disponible sera entièrement attribué à une institution poursuivant un but d'intérêt public analogue à celui de l'association et bénéficiant de l’exonération de l’impôt.

En aucun cas, les biens ne pourront retourner aux fondateurs physiques ou aux membres, ni être utilisés à leur profit en tout ou partie et de quelque manière que ce soit.

\chapter*{Entrée en vigueur}
\addcontentsline{toc}{chapter}{Entrée en vigueur}
Ces statuts ont été acceptés lors de l’Assemblée Générale du \underline{\hspace{3cm}} et entrent en vigueur immédiatement.

\vspace{1.5cm}

\makebox[0.7\textwidth]{Lieu, Date :\enspace\hrulefill}

\vspace{2cm}

\makebox[0.7\textwidth]{Le/a Président/e :\enspace\hrulefill}

\vspace{4cm}

\makebox[0.7\textwidth]{Le/a Secrétaire :\enspace\hrulefill}

\end{document}