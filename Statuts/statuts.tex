\documentclass[12pt,twoside]{report}

\usepackage{geometry}
 \geometry{
 a4paper,
 total={170mm,257mm},
 left=20mm,
 top=20mm,
 }

\usepackage[T1]{fontenc}
\usepackage{libertine}
\usepackage{enumitem}

\usepackage[utf8]{inputenc}
\usepackage[french]{babel}
\usepackage{graphicx}
\usepackage{hyperref}
\usepackage{lastpage}
\usepackage{fancyhdr}
\fancypagestyle{plain}{%
 \fancyhf{}%
 \fancyfoot{}%
 \fancyfoot[L]{The Swiss Privacy Basecamp}%
 \fancyfoot[C]{\thepage/\pageref*{LastPage}}%
 \fancyfoot[R]{EthACK.org — Statuts}%
 \renewcommand{\headrulewidth}{0pt}%
 \renewcommand{\footrulewidth}{0pt}%
}


\author{EthACK.org — The Swiss Privacy Basecamp}
\title{Statuts de l'Association}
{
\hypersetup{
	colorlinks=true,
	linkcolor=black,
	urlcolor=blue
}
\renewcommand{\theenumi}{\up{\arabic{enumi}}}
\renewcommand{\labelenumi}{\theenumi}


\begin{document}
\begin{titlepage}
\centering
\vspace{1cm}
\includegraphics[width=0.50\textwidth]{../logo-4096.png}\par\vspace{2cm}
{\scshape\LARGE EthACK.org \par}
\vspace{1cm}
{\scshape\Large The Swiss Privacy Basecamp \par}
\vspace{1.5cm}
{\huge\bfseries Statuts de l'Association\par}

\vfill
\end{titlepage}

\tableofcontents
\newpage


\setlength{\parindent}{0cm}

\section*{Dénomination et Siège}
\addcontentsline{toc}{chapter}{Dénomination et Siège}

\subsection*{Article 1}
\begin{enumerate}
\item EthACK.org est une Association sans but lucratif régie par les présents statuts et subsidiairement par les articles 60 et suivants du Code civil suisse\footnote{\url{http://www.admin.ch/ch/f/rs/210/index1.html}}. Elle est politiquement neutre et confessionnellement indépendante.
\end{enumerate}

\subsection*{Article 2}
\begin{enumerate}
\item Le siège de l’Association est situé à Vevey (Suisse).
\end{enumerate}

\section*{Buts}
\addcontentsline{toc}{chapter}{Buts}

\subsection*{Article 3}
\begin{enumerate}
\item L’Association poursuit les buts suivants :
\begin{itemize}
\item informer le public de leurs "droits numériques" via, par exemple, des articles en ligne, dans la presse ou des conférences ;
\item informer le public des possibles atteintes par des personnes privées ou des organes publics à ces droits via les mêmes moyens ;
\item promouvoir les outils servant à protéger sa sphère privée ;
\item former et sensibiliser le public pour protéger sa sphère privée ;
\item défendre les droits numériques actuels, et les renforcer ;
\end{itemize}
\end{enumerate}

\section*{Ressources}
\addcontentsline{toc}{chapter}{Ressources}

\subsection*{Article 4}
\begin{enumerate}
\item Les ressources de l’Association proviennent au besoin :
\begin{itemize}
\item de dons et legs ;
\item du parrainage ;
\item de subventions publiques ou privées ;
\item des cotisations versées par les membres ;
\item de toute autre ressource autorisée par la loi ;
\end{itemize}

\item Les comptes sont publiés après acceptation par l’Assemblée Générale. L’origine des revenus sera nominativement indiquée.
\end{enumerate}


\section*{Membres}
\addcontentsline{toc}{chapter}{Members}

\subsection*{Article 5}
\begin{enumerate}
\item Peut être membre toute personne physique ou morale intéressée par les buts de l’Association.
\end{enumerate}

\subsection*{Article 6}
\begin{enumerate}
\item Les membres se divisent en trois catégories :
\begin{itemize}
\item Membres Fondateurs ;
\item Membres Actifs ;
\item Membres de Soutien ;
\end{itemize}

\item La cotisation des Membres Fondateurs est fixée par l’Assemblée Générale.

\item Les Membres Actifs participent à la vie de l’Association par leur implication dans les différents buts définis à l’Article 3. La cotisation est fixée par l’Assemblée Générale.

\item La cotisation des Membres de Soutien est fixée par l’Assemblée Générale.
\end{enumerate}


\subsection*{Article 7}
\begin{enumerate}
\item Les demandes d’admission sont adressées au Comité. Le Comité admet ou refuse les nouveaux membres et en informe l’Assemblée Générale.
\end{enumerate}

\subsection*{Article 8}
\begin{enumerate}
\item La qualité de membre se perd :
\begin{itemize}
\item par la démission ;
\item par l’exclusion pour de justes motifs à la majorité du comité ;
\item par décès ;
\item par défaut de paiement des cotisations pendant plus d'une année ;
\end{itemize}

\item Hormis dans ce dernier cas, la cotisation de la période en cours reste due.

\item Un Membre Fondateur ne peut être exclu qu’à la majorité des Membres Fondateurs actifs dans l’Association.
\end{enumerate}

\subsection*{Article 9}
\begin{enumerate}
\item Le patrimoine de l'Association répond seul aux engagements contractés en son nom. Toute responsabilité personnelle de ses membres est exclue.
\end{enumerate}

\section*{Organes}
\addcontentsline{toc}{chapter}{Organes}

\subsection*{Article 10}
\begin{enumerate}
\item Les organes de l’Association sont :
\begin{itemize}
\item L’Assemblée Générale.
\item Le Comité.
\item L’organe de contrôle des comptes.
\end{itemize}

\item Tant qu’un Membre Fondateur est actif dans l’Association, au moins un Membre Fondateur doit être au Comité.
\end{enumerate}

\section*{Assemblée Générale}
\addcontentsline{toc}{chapter}{Assemblée Générale}

\subsection*{Article 11}
\begin{enumerate}
\item L’Assemblée Générale est le pouvoir suprême de l’Association. Elle est composée de tous les membres.

\item Elle se réunit une fois par an en session ordinaire. Elle peut, en outre, se réunir en session extraordinaire chaque fois que nécessaire à la demande de la majorité du Comité ou des Membres Fondateurs ou d’un cinquième des membres actifs.

\item L’Assemblée Générale est valablement constituée quel que soit le nombre de membres présents.

\item Le Comité communique aux membres par écrit la date de l'Assemblée générale au moins 3 semaines à l'avance. La convocation mentionnant l'ordre du jour est adressée par le Comité à chaque membre au moins 10 jours à l'avance.
\end{enumerate}

\subsection*{Article 12}
\begin{enumerate}
\item L’Assemblée Générale :
\begin{itemize}
\item élit les membres du Comité et désigne au moins un/e Président/e, un/e Secrétaire et un/e Trésorier/ère ;
\item prend connaissance des rapports et des comptes de l’exercice et vote leur approbation ;
\item approuve le budget annuel ;
\item fixe le montant des cotisations annuelles ;
\item prend position sur les autres sujets portés à l’ordre du jour ;
\item adopte et modifie les statuts ;
\end{itemize}
\end{enumerate}

\subsection*{Article 13}
\begin{enumerate}
\item Les décisions de l’Assemblée Générale sont prises à la majorité simple des membres présents. En cas d’égalité des voix, celle du président est prépondérante.
Les votations ont lieu à main levée. À la demande d’au moins cinq membres, elles auront lieu à bulletin secret.

\item Le vote par procuration n'est pas autorisé.

\item Les Membres de Soutien n’ont pas le droit de vote, et seuls les Membres Actifs à jour de cotisation peuvent voter.
\end{enumerate}

\subsection*{Article 14}
\begin{enumerate}
\item L’Assemblée Générale est présidée par le président ou un membre du Comité.
\end{enumerate}


\subsection*{Article 15}
\begin{enumerate}
\item L’ordre du jour de l’Assemblée Générale annuelle, dite "ordinaire", comprend nécessairement :
\begin{itemize}
\item l’approbation du procès-verbal de l’assemblée précédente ;
\item le rapport du Comité sur l’activité de l’Association durant la période écoulée ;
\item les rapports de trésorerie et de l’organe de contrôle des comptes ;
\item la fixation des cotisations ;
\item l’adoption du budget ;
\item l’approbation des rapports et comptes ;
\item l’élection des membres du Comité et de l’organe de contrôle des comptes ;
\item les propositions individuelles ;
\end{itemize}
\end{enumerate}

\section*{Comité}
\addcontentsline{toc}{chapter}{Comité}

\subsection*{Article 16}
\begin{enumerate}
%Le Comité est autorisé à faire tous les actes qui se rapportent au but de l'association. Il a les pouvoirs les plus étendus pour la gestion des affaires courantes.
\item Le Comité est compétent pour prendre toutes les décisions qui ne sont pas attribuées à un autre organe par les statuts.
\end{enumerate}

\subsection*{Article 17}
\begin{enumerate}
\item Le Comité se compose au minimum de 3 personnes élues par l’Assemblée Générale et choisies parmi les membres actifs.

\item La durée du mandat est de deux ans, renouvelable quatre fois. Les Membres Fondateurs n’ont pas de limite de mandat.

\item Il se réunit autant de fois que les affaires de l'association l'exigent.
\end{enumerate}

\subsection*{Article 18}
\begin{enumerate}
\item Les membres du comité agissent bénévolement.

\item Les éventuelles indemnisations sont régulées par le règlement interne du Comité.
\end{enumerate}

\subsection*{Article 19}
\begin{enumerate}
\item Le Comité est chargé :
\begin{itemize}
\item de prendre les mesures utiles pour atteindre les objectifs visés ; 
\item de convoquer les assemblées générales ordinaires et extraordinaires ; 
\item de prendre les décisions relatives à l’admission et à la démission des membres ainsi qu’à leur exclusion éventuelle ; 
\item de veiller à l’application des statuts, de rédiger les règlements et d’administrer les biens de l’Association ; 
\item de s'assurer de la bonne marche des moyens de communication de l'association ; 
\end{itemize}
\end{enumerate}

\subsection*{Article 20}
\begin{enumerate}
\item L’Association est valablement engagée par la signature collective de deux membres du Comité, dont le président.

\item Si ce dernier n'est pas disponible rapidement et qu'il y a péril en la demeure, deux autres membres du Comité peuvent signer ensemble, à charge pour eux d'informer le président dès que possible.
\end{enumerate}

\subsection*{Article 21}
\begin{enumerate}
\item Seuls le président, le secrétaire et la personne chargée des relations publiques sont habilités à représenter l'Association auprès du public et des médias.
\end{enumerate}

\section*{Dispositions diverses}
\addcontentsline{toc}{chapter}{Dispositions diverses}

\subsection*{Article 22}
\begin{enumerate}
\item L'exercice comptable commence le 1er janvier et se termine le 31 décembre de chaque année.

\item La gestion des comptes est confiée au trésorier de l'association et contrôlée chaque année par le(s) vérificateur(s) nommé(s) par l'Assemblée Générale.
\end{enumerate}

\subsection*{Article 23}
\begin{enumerate}
\item En cas de dissolution de l'association, l'actif disponible sera entièrement attribué à une institution poursuivant un but d'intérêt public analogue à celui de l'association et bénéficiant de l’exonération de l’impôt.

\item En aucun cas, les biens ne pourront retourner aux fondateurs ou aux membres, ni être utilisés à leur profit en tout ou partie et de quelque manière que ce soit.
\end{enumerate}


\section*{Entrée en vigueur}
\addcontentsline{toc}{chapter}{Entrée en vigueur}
Ces statuts ont été acceptés lors de l’Assemblée Générale du \underline{\hspace{3cm}} et entrent en vigueur immédiatement.

\vspace{1.5cm}

\makebox[0.7\textwidth]{Lieu, Date :\enspace\hrulefill}

\vspace{2cm}

\makebox[0.7\textwidth]{Le/a Président/e :\enspace\hrulefill}

\vspace{4cm}

\makebox[0.7\textwidth]{Le/a Secrétaire :\enspace\hrulefill}

\end{document}